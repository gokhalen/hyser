\def\b#1{\mbox{\boldmath{$#1$}}}
\documentclass[12pt]{article}
\usepackage{amsmath}
\usepackage{amsfonts}
\usepackage{graphicx}
\begin{document}
\title{Documentation for constitutive equation library routines}
\author{Nachiket Gokhale gokhalen@bu.edu}
\maketitle
\
\section{Introduction}
The constitutive library routines have facilities for the following relations:

\begin{enumerate}
\item{Non-linear elasticity in 1d,2d,3d}
\end{enumerate}

For each of these routines a data structure is defined in {\textit{\b{conslib.h}}}, which must be passed along with the desired constitutive equation number to the routines (for example, \textit{\b{elas1d}}) to get the value of the stiffness at a particular state. 
\section{Constitutive Equations in 2D}
\subsection{Notation}
Throughout this document we use the following notation:
\begin{eqnarray}
\b{F} &=& 1 + \nabla^{X}\b{U}(\b{X})\\
J &=& det(\b{F})\\
\b{C} &=& \b{F}^{T}\b{F}
\end{eqnarray}
\subsection{Law Number 1}
This is found on page 223 of Elastoplasticity and Viscoplasticity the unauthorized notes, by Simo and Hughes.
The strain energy function is 
\begin{eqnarray}
\mathcal{W} = \frac{\lambda}{4}(J^2 -1) - (\frac{\lambda}{2} + \mu)ln(J) + \frac{1}{2}\mu(tr(\b{C}) - 3)
\end{eqnarray}
The second-piola Kirchhoff is
\begin{eqnarray}
\b{S}  = \frac{\lambda}{2}(J^2-1)\b{C}^{-1} + \mu(\b{1} - \b{C}^{-1})
\end{eqnarray} 
The material stiffness is 
\begin{eqnarray}
C_{IJKL} = \frac{1}{2}(2\mu - \lambda(det(\b{C}) - 1))(C^{-1}_{IK}C^{-1}_{JL} + C^{-1}_{IL}C^{-1}_{JK}) + \lambda({det{\b{C}}})C^{-1}_{IJ}C^{-1}_{KL}
\end{eqnarray}
\end{document}
